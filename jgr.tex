\documentclass{article}
%\documentclass[jgrga]{agutex}

%\usepackage{natbib}
\usepackage{graphicx}
\usepackage{epstopdf}
\usepackage{times}
\usepackage{latexsym}
\usepackage{amssymb}
\usepackage[displaymath]{lineno}
\usepackage{indentfirst}
\usepackage{lineno}
\usepackage{lipsum} % For adding fake text.

\usepackage{units} % SI unit typesetting
\usepackage{xspace} % Automatically adjusting space after macros
\usepackage{amsmath} % \text, and other math formatting options
\usepackage{siunitx} % \num{} formatting and SI unit formatting

\usepackage{parskip} % http://ctan.org/pkg/parskip vskip instead of indent.

\usepackage[noabbrev,capitalize]{cleveref} % Automatically determine \cref type

\usepackage{xcolor} % so we can put todo notes in color.

% Configure the siunitx package
\sisetup{
    group-separator = {,}, % Use , to separate groups of digits, like 12,345
    list-final-separator = {, and } % Always use the serial comma in \SIlist
}

% Configure the cleveref package
\newcommand{\creflastconjunction}{, and } % Always use the serial comma

\linenumbers*[1]

%use to display figures in draft mode
\setkeys{Gin}{draft=false}

% ######################################################################
% ################################################### Shorthand / Macros
% ######################################################################

\newcommand{\about}{\ensuremath{\sim}}

\newcommand{\real}{\ensuremath{\mathbb{R}\mathrm{e}}\xspace}
\newcommand{\imag}{\ensuremath{\mathbb{I}\mathrm{m}}\xspace}

\newcommand{\DST}{\text{Dst}\xspace}

\newcommand{\dft}[1]{\ensuremath{\overset{\sim}{#1}}\xspace}

% show todo notes in red.
\newcommand{\todo}[1]{ \textcolor{red}{TODO: #1} }
% hide todo notes.
%\newcommand{\todo}[1]{}

% Names with special characters.
\newcommand{\Alfven}{Alfv\'en\xspace}
\newcommand{\Ampere}{Amp\`ere\xspace}
\newcommand{\Schrodinger}{Schr\"odinger\xspace}

% To make sure the capitalization is consistent.
\newcommand{\ohmlaw}{Ohm's law\xspace}
\newcommand{\amplaw}{\Ampere's law\xspace}
\newcommand{\farlaw}{Faraday's law\xspace}
\newcommand{\maxeqs}{Maxwell's equations\xspace}

% What should Radoski's dipole coordinates be named? \nu is overused.
\newcommand{\radx}{\ensuremath{x}\xspace}
\newcommand{\rady}{\ensuremath{y}\xspace}
\newcommand{\radz}{\ensuremath{z}\xspace}

% What should Lysak's coordinates be named? I don't love u1 u2 u3.
\newcommand{\lysaki}{\ensuremath{u^i}\xspace}
\newcommand{\lysakj}{\ensuremath{u^j}\xspace}
\newcommand{\lysakx}{\ensuremath{u^1}\xspace}
\newcommand{\lysaky}{\ensuremath{u^2}\xspace}
\newcommand{\lysakz}{\ensuremath{u^3}\xspace}

% Coordinate names... these should probably be italicized?
\newcommand{\x}{\ensuremath{x}\xspace}
\newcommand{\y}{\ensuremath{y}\xspace}
\newcommand{\z}{\ensuremath{z}\xspace}
\newcommand{\X}{\ensuremath{X}\xspace}
\newcommand{\Y}{\ensuremath{Y}\xspace}
\newcommand{\Z}{\ensuremath{Z}\xspace}

% Field-aligned unit vectors.
\newcommand{\ehat}{\ensuremath{\hat{e}}\xspace}
\newcommand{\xhat}{\ensuremath{\hat{x}}\xspace}
\newcommand{\yhat}{\ensuremath{\hat{y}}\xspace}
\newcommand{\zhat}{\ensuremath{\hat{z}}\xspace}
\newcommand{\Xhat}{\ensuremath{\hat{X}}\xspace}
\newcommand{\Yhat}{\ensuremath{\hat{Y}}\xspace}
\newcommand{\Zhat}{\ensuremath{\hat{Z}}\xspace}

% Spherical unit vectors.
\newcommand{\rhat}{\ensuremath{\hat{r}}\xspace}
\newcommand{\qhat}{\ensuremath{\hat{\theta}}\xspace}
\newcommand{\fhat}{\ensuremath{\hat{\phi}}\xspace}

% Use underlines for vectors and tensors.
\renewcommand{\vec}[1]{\ensuremath{\underline{#1}}}
\newcommand{\tensor}[1]{\ensuremath{\underline{\underline{#1}}}}

% Differential operators.
\newcommand{\dd}[1]{\ensuremath{ \frac{\partial}{\partial #1} }\xspace}
\newcommand{\ddt}{\dd{t}\xspace}
\newcommand{\curl}[1]{\ensuremath{ \nabla \times \vec{#1} }\xspace}
\renewcommand{\div}[1]{\ensuremath{ \nabla \cdot \vec{#1} }\xspace}
\newcommand{\grad}[1]{\ensuremath{ \nabla #1 }\xspace}

% Properly-scaled parentheses for grouping terms or for arguments.
\newcommand{\lr}[1]{ \left( #1 \right) }
\newcommand{\lrsmall}[1]{ \left( {\scriptstyle #1} \right) }
\renewcommand{\arg}[1]{\!\lr{#1}}
\newcommand{\argsmall}[1]{\!\lrsmall{#1}}
\newcommand{\lrb}[1]{ \left[ #1 \right] }
\newcommand{\argb}[1]{\!\lrb{#1}}

% Define a better looking eV by moving the V slightly left
\DeclareSIUnit\electronvolt{e\hspace{-0.08em}V}
\DeclareSIUnit\keV{\kilo\electronvolt}
\DeclareSIUnit\percc{/\cm\cubed}
\DeclareSIUnit\RE{R_E}
\DeclareSIUnit\nT{\nano\tesla}
\DeclareSIUnit\nJ{\nano\joule}
\DeclareSIUnit\S{S}
\DeclareSIUnit\Mm{Mm}

\newcommand{\dt}{\ensuremath{\delta \hspace{-0.1em} t}\xspace}
\newcommand{\dr}{\ensuremath{\delta \hspace{-0.1em} r}\xspace}
\newcommand{\dL}{\ensuremath{\delta \hspace{-0.1em} L}\xspace}

% Assignment operator for pseudocode.
\newcommand{\assign}{\ensuremath{\leftarrow}\xspace}

% Azimuthal modenumber, typically indicated with a lowercase m.
\newcommand{\azm}{\ensuremath{m}\xspace}

% Electron mass, also typically indicated with a lowercase m.
\newcommand{\me}{\ensuremath{m_{e}}\xspace}

% Jacobian dererminant, typically indicated with a capital J, which we are using for current.
\newcommand{\jac}{\ensuremath{ \sqrt{g} }\xspace}

% Plasma frequency
\newcommand{\op}{\ensuremath{\omega_P}\xspace}

% Alfven speed.
\newcommand{\va}{\ensuremath{v_A}\xspace}

% Perpendicular electric constant.
\newcommand{\ep}{\ensuremath{\epsilon_\bot}\xspace}

% Epsilon zero, mu zero, and one over mu zero.
\newcommand{\ez}{\ensuremath{\epsilon_0}\xspace}
\newcommand{\mz}{\ensuremath{\mu_0}\xspace}
\newcommand{\oomz}{\ensuremath{ \frac{1}{\mz} }\xspace}

% Conductivities.
\newcommand{\sz}{\ensuremath{\sigma_0}\xspace}
\newcommand{\sh}{\ensuremath{\sigma_H}\xspace}
\renewcommand{\sp}{\ensuremath{\sigma_P}\xspace}

% Speed of light.
\newcommand{\C}{{\mathrm{c}}}

% Add space between rows of tables
\newcommand{\spacerows}[1]{\renewcommand{\arraystretch}{#1}}

% ######################################################################
% ################################################# Title, Abstract, etc
% ######################################################################

\begin{document}

\title{Modeling Pc4 Pulsations in Two and a Half Dimensions with Comparisons to Van Allen Probes Observations}

\author{
    Charles McEachern\textsuperscript{1},
    Robert Lysak\textsuperscript{1},
    ... \\
    \textsuperscript{1} University of Minnesota
}

%\affil{University of Minnesota}

%\lefthead{McEachern et al.}
%\righthead{Pc4s in 2.5D}
%\linespread{2}

%\linenumbers

\maketitle

% ======================================================================
% ============================================================= Abstract
% ======================================================================

\begin{abstract}

Field line resonances in the Pc4 range (\SIrange{7}{25}{\mHz}) serve to energize magnetospheric particles through drift-resonant interactions, carry energy from high to low altitude, induce currents in the magnetosphere, and accelerate particles into the atmosphere. Wave structure and polarization significantly impact the execution these roles. The present work showcases a new two and a half dimensional code, Tuna, ideally suited to model FLRs, with the ability to consider large-but-finite azimuthal modenumbers, coupling between the poloidal, toroidal, and compressional modes, and arbitrary harmonic structure. Using Tuna, the interplay between Joule dissipation and poloidal-to-toroidal rotation is considered for Pc4 pulsations under both dayside and nightside conditions. An attempt is also made to demystify giant pulsations, a class of Pc4 noted for its distinctive ground signatures. Numerical results are supplemented by a survey of \about700 Pc4 pulsations using data from the Van Allen Probes, the first such survey to characterize each event by both polarization and harmonic. The combination of numerical and observational results suggests an explanation for the disparate distributions observed in poloidal and toroidal Pc4 events.

\end{abstract}

%\begin{article}

% ######################################################################
% ######################################################### Introduction
% ######################################################################

\section{Introduction}

Pc4s are interesting.

\begin{itemize}
    \item Drift-resonant interactions with trapped energetic particles.
    \item Radial diffusion.
    \item Giant pulsations are mysterious and exciting.
\end{itemize}

Pc4s can be classified in terms of their harmonic numbers.

\begin{itemize}
    \item First harmonic is for drift resonance; second harmonic is for drift-bounce resonance\cite{dai_2013,poulter_1983}.
    \item You can sorta tell them apart by frequency, but there are disagreements\cite{takahashi_2013}. Mostly you need two observations -- satellite plus ground, or E and B on the same machine\cite{dai_2015}, like THEMIS\cite{angelopoulos_2008} or the Van Allen Probes\cite{stratton_2012}.
\end{itemize}

They can also be classified in terms of their azimuthal modenumber.

\begin{itemize}
    \item One wavelength of a wave with modenumber \azm spans $\frac{24}{\azm}$ hours in MLT.
    \item Small \azm typically driven by broadband solar wind behavior\cite{degeling_2014,hao_2014,zong_2009,chen_1974,liu_2011,southwood_1974}. The waves are compressional -- they can move across magnetic field lines.
    \item Large \azm is driven by resonant interactions with trapped particles -- the waves are not compressional, so propagation is evanescent across field lines\cite{cummings_1969,radoski_1974}.
    \item Ground observations depend on \azm. At large \azm, a wave's ground signature is attenuated by the ionosphere\cite{hughes_1976,wright_1999,yeoman_2001}.
\end{itemize}

They can also be classified in terms of their polarization.

\begin{itemize}
    \item Poloidal waves pulse. Magnetic field lines perturb radially. Associated azimuthal electric field perturbations.
    \item Toroidal waves twist. Magnetic field lines perturb azimuthally. Associated radial electric field perturbations.
    \item Poloidal waves rotate asymptotically to the toroidal mode. High-\azm poloidal waves have shorter lifetimes\cite{mann_1995,mann_1997,radoski_1974}.
    \item Most observed waves are toroidal\cite{anderson_1990}.
    \item Most poloidal waves are even modes\cite{hughes_1978,singer_1982,takahashi_1990}.
    \item Toroidal frequencies are sharply defined in L. Poloidal show a more smeared-out dependence. \cite{engebretson_1986}
    \item In this regime, wave polarization is rotated about \SI{90}{\degree} by the ionosphere\cite{nishida_1964_screening}. An east-west magnetic perturbation in space is observed as a north-south perturbation at Earth's surface.
\end{itemize}

Let's also talk about giant pulsations.

\begin{itemize}
    \item Strong, well-formed ground signatures. Tabulated by eye for a century\cite{birkeland_1901}.
    \item Harmonic structure was a point of contention for decades, but recent multisatellite observations seem to be in agreement that they are odd harmonics, probably fundamental\cite{glassmeier_1999,hillebrand_1982,kokubun_1989,takahashi_2011}.
    \item Poloidal Pc4s, peaked from postmidnight to the early morning\cite{chisham_1991,glassmeier_1980,rostoker_1979}.
    \item A distinct break from mostly-toroidal, mostly-dayside Pc4s in general.
    \item Most common during times of low solar activity\cite{brekke_1987}.
    \item Chirality flips based on latitude, even within a single event\cite{eleman_1967}.
\end{itemize}

Past models of the magnetosphere have been limited in their consideration of FLRs. Reasons include overly-simplified treatment of the ionospheric boundary, no consideration of the plasmapause, limited range in \azm, and the inability to compute ground signatures. The present work showcases a model which addresses these issues, providing a bird’s-eye view of the structure and evolution of FLRs.

Using this model, the present work explores novel connections between several of the seemingly-disparate Pc4 properties listed above. Poloidal-to-toroidal rotation timescales are compared to dayside and nightside Joule dissipation timescales; the implications to Pc4 observations are considered. The strength and structure of ground signatures is investigated as function of ionospheric and driving conditions. And the distinctive properties associated with giant pulsations are matched against those of fundamental poloidal Pc4s in general.

Results are then validated against a survey of \about700 Pc4 observations using data collected by the Van Allen probes. In contrast to other ULF wave surveys (for example, \cite{dai_2015} and \cite{motoba_2015}), the present work classifies each event by both polarization and harmonic. This crucial aspect of the analysis is possible only because the Van Allen Probes measure both electric and magnetic field waveforms. No past mission has provided access to such a rich data set for ULF wave events in the inner magnetosphere.

% ######################################################################
% ################################################################ Model
% ######################################################################

\section{Numerical Model}

\todo{You can get this code on GitHub!}

Numerical results are obtained using Tuna, a new linear \Alfven wave code based on that described in \cite{lysak_2013}. Tuna models the evolution of three-dimensional electric and magnetic fields over a (two-dimensional) meridional slice of the magnetosphere. The code can colloquially be said to have two and a half (``tuna half'') dimensions, hence the name.

For the purpose of evaluating derivatives in the azimuthal direction, fields are taken to vary as $\exp \arg{i \azm \phi}$ for fixed azimuthal modenumber \azm. Azimuthal derivatives are replaced by a factor of $i \azm$; fields are complex-valued as a result. This assumption is easily justified in the case of Pc4 pulsations, which are typically localized in MLT, per\cite{anderson_1990,dai_2015,engebretson_1992,liu_2009}.

% ======================================================================
% ========================================== Physical Parameter Profiles
% ======================================================================

Empirical profiles are used for the conductivity tensor $\tensor{\sigma}$ and the electric tensor $\tensor{\epsilon}$. The conductivity tensor $\tensor{\sigma}$ comes from values tabulated in \cite{kelley_1989}, and modified per \cite{lysak_2013} to take into account the loading of oxygen ions near the atmosphere. The electric tensor $\tensor{\epsilon}$ gives characteristic velocity $c$ along the magnetic field line and $\va$ in the perpendicular direction, where the \Alfven speed $\va$ is defined per
\begin{align}
    \label{def_va}
    \va^2 &\equiv \frac{B^2}{\rho} &
    & \text{or, equally,} &
    \va^2 &\equiv \frac{1}{\mz\ep}
\end{align}

\todo{Figure of conductivity profiles? Dissertation figure 5.3}

In \cref{def_va}, $B$ is the magnitude of the zeroth-order magnetic field, taken to be an ideal dipole field with magnitude \SI{3.1e4}{\nT} at the equator at Earth's surface. The density profile is modeled as the sum of a latitude-independent profile (\SI{10}{\percc} at the ionosphere, falling off as $\frac{1}{r}$) and a latitude-dependent one (\SI{e4}{\percc} at the ionosphere, with a sharp drop at $L = 4$).

Four different physical parameter profiles are used for conductivity and \Alfven speed, corresponding to dayside and nightside conditions at the top and bottom of the solar cycle.

% ======================================================================
% ============================================ Nonorthogonal Dipole Grid
% ======================================================================

Tuna's grid follows the modified dipole coordinates described in \cite{lysak_2004}:
\begin{align}
  \label{def_coords}
  \lysakx & = - \frac{R}{r} \sin^2 \theta &
  \lysaky & = \phi &
  \lysakz & = \frac{R^2}{r^2} \frac{\cos \theta}{\cos \theta_0}
\end{align}

Here, $R$ is the geocentric radius of the ionospheric boundary, taken to be at $R_E + \SI{100}{\km}$, $\theta_0$ is the invariant colatitude, and $r$, $\theta$, and $\phi$ are the usual spherical coordinates.

\todo{Figure of the grid? Dissertation figure 5.1}

A thorough discussion of these coordinates, and explicit forms for the resulting basis vectors and metric tensor components can be found in \cite{lysak_2004}. At present, it's sufficient to note that the coordinates are nonorthogonal, and thus have covariant and contravariant basis vectors (${\ehat_i \equiv \dd{\lysaki}\vec{r}}$ and ${\ehat^i \equiv \dd{\vec{r}}\lysaki}$ respectively) that do not line up with one another --- but that both the covariant and contravariant bases are valuable.

The basis vectors $\ehat^1$, $\ehat^2$, and $\ehat_3$ provide a mapping to the usual dipole coordinates, as shown in \cref{to_dipole}, which is the natural basis for the conductivity and electric tensors.
\begin{align}
    \label{to_dipole}
    \ehat^1 &\parallel \xhat &
    \ehat^2 &\parallel \yhat &
    \ehat_3 &\parallel \zhat
\end{align}

Where $\zhat$ lies along the magnetic field, $\yhat$ is azimuthally eastward, and $\xhat \equiv \yhat \times \zhat$ points radially outward at the equator.

In addition, at the ionospheric boundary, $\ehat_1$, $\ehat_2$, and $\ehat^3$ map to the spherical basis:
\begin{align}
  \ehat_1 &\parallel \hat{\theta} &
  \ehat_2 &\parallel \hat{\phi} &
  \ehat^3 &\parallel \hat{r}
\end{align}

As a result, Tuna's grid is aligned everywhere to the zeroth-order dipole magnetic field, while also supporting a fixed-altitude ionospheric boundary.

The results shown in the present work use a grid of 150 values in \lysakx (150 field lines) and 350 values in \lysakz (350 grid points per field line). Spacing is on the order of \SI{10}{\km} near the ionosphere and \SI{1000}{\km} at the equator of the outermost field line. The inner boundary is placed at $L = 2$, and the outer boundary at $L = 10$. The time step is determined from the smallest \Alfven crossing time, scaled down by a Courant factor of \num{0.1}. Typically, $\dt \about \SI{10}{\us}$.

% ======================================================================
% ================================================= Ring Current Driving
% ======================================================================

Like the similar models of \cite{lysak_2013} and \cite{waters_2013}, Tuna can be driven via compression of the simulation's outer boundary -- typically taken as a proxy for solar wind activity. However, the shear and compressional \Alfven modes decouple at high \azm, preventing such waves from propagating across magnetic field lines. In order to model noncompressional Pc4 activity at $L\about5$, it's necessary to inject energy at $L\about5$.

To this end, Tuna also allows simulations to be driven via modulation of the ring current, a stand-in for substorm injection events. The driving current is spread over a cross section of \about\SI{1}{\RE}$^2$, centered just outside the plasmapause and just off the equator. In effect, the energy is delivered into a first-harmonic poloidal wave.

The magnitude of the driving current is estimated from a discrete Fourier transform of the Sym-H storm index during the June 2013 storm. Sym-H is tabulated once per minute, which is too slow to capture activity in the Pc4 band directly. However, a fit of the pink noise suggests that activity in the Pc4 range could plausibly give rise to a field at Earth's surface on the order of \SI{e-2}{\nT}.

This corresponds to a ring current perturbation on the order of \SI{1}{\mega\A}, or \about\SI{e-4}{\uA/\m\squared} spread over \about\SI{1}{\RE}$^2$. For the runs shown in the present work, the driving current is azimuthally directed, and its magnitude varies sinusoidally.

% ======================================================================
% ================================================== Maxwell's Equations
% ======================================================================

Mathematically, the driving current is introduced through an anomalous current term $\vec{J}$ in \Ampere's law, apart from the Ohmic current ${\tensor{\sigma} \cdot \vec{E}}$.
\begin{align}
    \label{amp_law}
    \tensor{\epsilon} \cdot \ddt \vec{E} &= \oomz \curl{B} - \vec{J}
      - \tensor{\sigma} \cdot \vec{E}
\end{align}

Because of the Ohmic current term, the electric field's time derivative in \cref{amp_law} depends on its own future value. This circular dependence is resolved using integrating factors. First, the expression is rewritten:
\begin{align}
    \label{int_fac}
    \Big( \tensor{\Omega} + \tensor{ \mathbb{I} }\ddt \Big) \cdot
        \vec{E} &= \tensor{V}^2 \cdot \vec{F}
\end{align}

Where $\tensor{ \mathbb{I} }$ is the identity and $\vec{F}$, $\tensor{V}^2$, and $\tensor{\Omega}$ are shorthand:
\begin{align}
    \vec{F} &\equiv \curl{B} - \mz \vec{J} &
    \tensor{V}^2 &\equiv \frac{1}{\mz} \tensor{\epsilon}^{-1} &
    \tensor{\Omega} &\equiv \tensor{\epsilon}^{-1} \cdot \tensor{\sigma}
\end{align}

\cref{int_fac} is then solved by multiplying through by $\exp \arg{ \tensor{\Omega} \, t }$ (see \cite{hall_2015}), applying the product rule, and integrating over a time step \dt.
\begin{align}
    \label{amp_final}
    \vec{E} &\assign \exp \arg{ -\tensor{\Omega} \, \dt } \cdot \vec{E} +
        \dt \, \exp \arg{ -\tensor{\Omega} \, \tfrac{\dt}{2} } \cdot
        \tensor{V}^2 \cdot \vec{F}
\end{align}

\cref{amp_final} is evaluated by separating the exponential into its diagonal (Pedersen) and off-diagonal (Hall) terms. The Hall terms give a rotation matrix around the magnetic field line, coupling the poloidal and toroidal modes, consistent with \cite{hughes_1974}. Terms proportional to $\exp \arg{ - \frac{\sz}{\ez}\dt }$ are also present. However, $\frac{\sz}{\ez}\dt \gtrsim 1000$, so these terms are vanishingly small. As a result, parallel electric fields are taken to be uniformly zero.

Magnetic fields are simply advanced using Faraday's law:
\begin{align}
    \label{far_law}
    \ddt \vec{B} &= - \curl{E}
\end{align}

For the sake of brevity, the present work does not expand the terms of \cref{amp_final,far_law} in the covariant basis. Those expressions can be found in \cite{mceachern_2016}.

% ======================================================================
% ================================================== Boundary Conditions
% ======================================================================

Dirichlet and Neumann conditions are applied to the electric and magnetic fields respectively at the inner and outer boundaries. Results of the present work are robust under an exchange of the two.

Between the top of the neutral atmosphere and the bottom of the ionosphere, the model includes a thin, horizontal current sheet representing the ionosphere's $E$ layer\cite{lysak_2004}. By integrating \amplaw over the layer, it can be shown\cite{fujita_1988} that the horizontal electric field values at the edge of the grid are determined by the jump in the horizontal magnetic fields:
\begin{align}
  \label{jump_condition}
  \tensor{\Sigma} \cdot \vec{E} &= \frac{1}{\mz} \,
    \displaystyle\lim_{\dr \rightarrow 0} \, \bigg[ \, \hat{r} \times \vec{B}
    \, \bigg|^{R_I + \dr}_{R_I - \dr}
\end{align}

The atmospheric magnetic field is computed in terms of a scalar magnetic potential, $\Psi$, such that $\vec{B}=\grad{\Psi}$. The neutral atmosphere is taken to be a perfect insulator, giving $\curl{B}=0$. Combined with $\div{B}=0$ (per Maxwell's equations), this ensures that $\Psi$ satisfies Laplace's equation, $\nabla^2 \Psi = 0$, and thus can be written as a sum of harmonics.
\begin{align}
  \label{psi_expansion}
  \Psi &= \displaystyle\sum_\ell \lr{ a_\ell \, r^\ell +
    b_\ell \, r^{-\ell - 1} } Y_\ell
\end{align}

Earth is taken to be a perfect conductor, so $B_r = \dd{r} \Psi = 0$ at $R_E$. In addition, the thin current sheet at the top of the atmosphere is taken to be horizontal, so the radial component of the magnetic field must be the same just above and just below it. Those two boundary conditions (combined with the harmonics' orthonormality) allow solutions for the coefficients $a_\ell$ and $b_\ell$, giving the following for $\Psi$ at the bottom and top of the atmosphere respectively:
\begin{align}
  \label{psi_final}
  \begin{split}
  \Psi_E &= \displaystyle\sum_\ell Y_\ell \; \frac{R_I}{ \ell \, \lr{\ell - 1} } \frac{ \lr{2 \ell - 1} \, \lambda^\ell }{ 1 - \lambda^{2 \ell + 1} } B_r \cdot Y_\ell^{-1} \\
  \Psi_I &= \displaystyle\sum_\ell Y_\ell \; \frac{R_I}{ \ell \, \lr{\ell - 1} } \frac{ \lr{\ell - 1} + \ell \, \lambda^{2 \ell + 1} }{ 1 - \lambda^{2 \ell + 1} } B_r \cdot Y_\ell^{-1}
  \end{split}
\end{align}

Where $\lambda \equiv \frac{R_E}{R_I} \sim \num{0.975}$ and $B_r \cdot Y_\ell^{-1} \equiv \displaystyle\sum_i B_r [i] \; Y_\ell^{-1} \! [i]$.

Whereas magnetic field values at the top of the atmosphere are used to compute boundary electric field values, those at the bottom of the atmosphere are purely output, suitable for comparison with magnetometer data.

% ######################################################################
% #################################################### Numerical Results
% ######################################################################

\section{Numerical Results}

See \cref{fig_brms}

\begin{itemize}
    \item At low \azm, all components of the magnetic field are comparable in magnitude.
    \item At high \azm, the compressional component weakens.
    \item Toroidal activity is sharply concentrated at resonant $L$.
    \item Poloidal activity is spread out in $L$. At high \azm, it just can't spread very far.
\end{itemize}

\begin{figure}
    \label{fig_brms}
    \begin{center}
    \includegraphics[width=\textwidth]{figures/fig_brms.pdf}
    \caption{
        Each cell in the above figure shows root-mean-square magnetic field perturbations over the course of a \SI{300}{\s} run. The four columns show four different runs, with azimuthal modenumbers 1, 4, 16, and 64 respectively. The rows show poloidal, toroidal, and compressional magnetic field components. At low \azm, compressional activity is apparent by the fact that all three components of the magnetic perturbation are comparable in magnitude; as \azm increases, compressional activity diminishes. Poloidal waves are spread broadly in $L$ at low \azm, becoming guided only as large \azm prevents energy from moving across field lines. In contrast, toroidal waves are sharply defined in $L$ regardless of \azm.
    }
    \end{center}
\end{figure}

% ----------------------------------------------------------------------

See \cref{fig_energy}

\begin{itemize}
    \item This figure mimics Figure 3 in \cite{mann_1995}
    \item All driving is poloidal. That any energy is in the toroidal mode shows rotation.
    \item The rotation seems to be one-way, as Mann suggested. High-\azm waves rotate slower, also consistent with Mann.
    \item Note that Mann used perfectly conducting ionospheres.
    \item On the nightside, dissipation is as fast as rotation. There is no accumulation of energy over multiple drive periods.
    \item Poloidal mode looks to be a significant source of (same-harmonic) toroidal waves on the dayside. On the nightside, energy dissipates too fast to rotate to the toroidal mode.
\end{itemize}

\begin{figure}
    \label{fig_energy}
    \begin{center}
    \includegraphics[width=\textwidth]{figures/fig_energy.pdf}
    \caption{
        Each cell above shows integrated poloidal (blue) and toroidal (red) energy as a function of time for a single run. The top row shows runs using a dayside ionospheric profile (driven at \SI{22}{\mHz}), and the bottom row nightside (driven at \SI{16}{\mHz}). Driving is purely poloidal, and energy rotates over time to the toroidal mode. On the dayside, it's clear that energy rotates faster when \azm is smaller, consistent with the findings of \cite{mann_1995}. On the nightside, the dissipation timescale is comparable to the rotation timescale, so there is no long-term accumulation of energy in either mode.
    }
    \end{center}
\end{figure}

% ----------------------------------------------------------------------

See \cref{fig_layers_day}

\begin{itemize}
    \item At low \azm, energy moves freely across field lines, even escaping the simulation domain. This makes it difficult to build up a string toroidal resonance in any particular place.
    \item That clump at the top is probably nonphysical -- a third harmonic very close to the simulation boundary.
    \item As \azm increases, energy is less able to cross field lines. Maximum \azm, where the poloidal mode deposits all its energy on the same field line, is where the toroidal mode is strongest.
    \item (Would require an additional plot to show) If the drive frequency does not line up with the local eigenfrequency at high \azm, energy can't move across field lines to find a better match. Instead, the resonance is just weaker.
\end{itemize}

\begin{figure}
    \label{fig_layers_day}
    \begin{center}
    \includegraphics[width=\textwidth]{figures/fig_layers_day.pdf}
    \caption{
        Above are the same runs shown in the top row of \cref{fig_energy}; the runs are the same except that \azm increases to the right. Rather than show the energy integrated over the whole simulation domain, the above figure shows energy density, with time on the horizontal axis and $L$ on the vertical axis. The top and bottom rows show poloidal and toroidal energy distributions respectively. Toroidal activity is shown to be sharply concentrated at resonant $L$ shells; it's strongest at high \azm because that's where energy rotates most effectively from the poloidal mode. Poloidal waves are spread broadly in $L$. At low \azm, some energy escapes the simulation domain entirely. Poloidal waves appear sharp only when high \azm prevents energy from spreading.
    }
    \end{center}
\end{figure}

% ----------------------------------------------------------------------

See \cref{fig_layers_night}

\begin{itemize}
    \item At low \azm, as on the dayside, energy moves freely across field lines and even escapes.
    \item At high \azm, poloidal-to-toroidal rotation is slow compared to dissipation timescales. This is not the case on the dayside.
    \item As a result, the strongest nightside toroidal activity is at moderate \azm -- but overall the poloidal mode is a worse toroidal source on the nightside than it is on the dayside.
\end{itemize}

\begin{figure}
    \label{fig_layers_night}
    \begin{center}
    \includegraphics[width=\textwidth]{figures/fig_layers_night.pdf}
    \caption{
        The above figure is analogous to \cref{fig_layers_night}, except that it shows runs carried out using a nightside ionospheric profile instead of dayside. Four runs are shown, each at a different \azm, and the rows show poloidal and toroidal energy density as a function of $L$ and time. In a general sense, the behavior matches that seen on the dayside: poloidal activity is broad in $L$ while toroidal waves appear only where resonant. The difference is in the wave magnitude. With dayside and nightside driving at the same magnitude, the response on the nightside is smaller by an order of magnitude due to the increased Joule dissipation.
    }
    \end{center}
\end{figure}

% ----------------------------------------------------------------------


See \cref{fig_ground_day,fig_ground_night}

\begin{itemize}
    \item At low \azm, ground signatures are weak because waves in space are weak.
    \item At high \azm, ground signatures are weak due to attenuation by the atmosphere.
    \item Ground signatures seem to be strongest at \azm of 16 to 32.
    \item Observers on the top and bottom halves of the event will see opposite chiralities. This is clearest on the nightside events.
    \item The strongest peaks are in $B_\phi$, which corresponds to the poloidal mode.
    \item These properties match the usual description of a giant pulsation.
\end{itemize}

\begin{figure}
    \label{fig_ground_day}
    \begin{center}
    \includegraphics[width=\textwidth]{figures/fig_ground_day.pdf}
    \caption{
        The magnetic ground signatures are shown for the same runs as in previous figures. Ground signatures at low \azm are weak because the waves in space are weak. Ground signatures at high \azm are attenuated before reaching ground magnetometers. The two effects combine to create a maximum at \azm\about16. The maximum ground signatures consistently appear in the east-west magnetic field component, corresponding to the poloidal mode. Careful examination shows furthermore that the events are clockwise to an observer above \about\SI{65}{\degree} and counterclockwise to an observer below. All of these properties are commonly ascribed to giant pulsations.
    }
    \end{center}
\end{figure}

\begin{figure}
    \label{fig_ground_night}
    \begin{center}
    \includegraphics[width=\textwidth]{figures/fig_ground_night.pdf}
    \caption{
        Above are the ground signatures for the four nightside runs. As on the dayside, shown in \cref{fig_ground_day}, magnetic fields at Earth's surface are strongest when \azm\about16, and peak signals correspond to the poloidal mode. Compared to the dayside, ground signatures on the nightside are weaker, and the chirality shift (clockwise to the north and counterclockwise to the south) is much clearer.
    }
    \end{center}
\end{figure}

% ----------------------------------------------------------------------

The findings together suggest that the morphology of giant pulsations reveals relatively little about their origins.

One can consider a hypothetical magnetosphere subject to constant driving: broadband in frequency, broadband in modenumber, just outside the plasmapause. Low-\azm poloidal waves will quickly rotate to the toroidal mode (and/or propagate away). High-\azm waves will resonate in place, accumulating energy over time, and giving rise to ``multiharmonic toroidal waves'' (per \cite{takahashi_2011}); Fourier components that do not match the local eigenfrequency will accumulate energy over just a few wave periods before reaching asymptotic values. Waves with very high modenumbers will be attenuated by the ionosphere. The response on the ground will be counterclockwise at low latitude, clockwise at high latitude, peaked at modenumbers of 16 to 32, and mostly east-west polarized. In other words, the measurements will look very much like a giant pulsation.

The present results offer no explanation as to the tendency of giant pulsations to drift azimuthally, or to appear pre-dawn in MLT --- though the latter is addressed by the observational results in below.

% ######################################################################
% ##################################################### Van Allen Probes
% ######################################################################

\section{Van Allen Probes Observations}

\todo{Figure: A sample event.}

The present chapter gives a survey of \about700 thirty-minute Pc4 events, each characterized in terms of both parity and polarization, and selected in a way that does not introduce an apparent bias in either property. No past study has so thoroughly disentangled the parity and polarization of these waves.

\begin{figure}
    \label{fig_pos}
    \begin{center}
    \includegraphics[width=\textwidth]{figures/fig_pos.pdf}
    \caption{
        The above figure shows the distribution of data collected by the Van Allen Probes for which 3D electric and magnetic fields are available. (When the probe's axis aligns too closely with the magnetic field, the 3D electric field cannot be determined reliably.) Coverage is lopsided because the probes had completed one and a half precessions around Earth; that is, the nightside is double-sampled at apogee. Coverage is good outside $L\about4$; note that each day of sampling is broken down into 48 half-hour events.
    }
    \end{center}
\end{figure}

Coarsely speaking, event distributions are found to be consistent with past surveys. Toroidal events dominate overall, and are primarily seen on the morning side. Poloidal events are spread broadly in MLT, with a peak near noon and distinctive odd harmonics in the early morning. From there, the simultaneous consideration of harmonic and polarization, combined with the numerical results above, offers significant insight.

\begin{figure}
    \label{fig_all}
    \begin{center}
    \includegraphics[width=\textwidth]{figures/fig_all.pdf}
    \caption{
        ...
    }
    \end{center}
\end{figure}

The near-noon peak of poloidal Pc4 events is shown to be due to even events (a majority subset). Odd poloidal events occur preferrentially near midnight and across the morning side. Similarly, toroidal events are mostly odd, and it is specifically the odd toroidal events which occur on the morningside, while even toroidal events peak near noon.

The spatial distribution of even poloidal events looks much like the spatial distribution of even toroidal events, except that the toroidal distribution is skewed dayward compared to the poloidal. The same can be said of the odd events. This is consistent (per the above numerical results) with poloidal events as an effective source for (same-parity) toroidal events on the dayside, and a less-effective source on the nightside.

\begin{figure}
    \label{fig_mode}
    \begin{center}
    \includegraphics[width=\textwidth]{figures/fig_mode.pdf}
    \caption{
        ...
    }
    \end{center}
\end{figure}

As a corrolary, the distribution of odd poloidal events is found to closely resemble the distribution of giant pulsations: midnight and morning. This (consistent with the numerical results) suggests that the distinctive properties attributed to giant pulsations are in fact shared by odd poloidal Pc4s overall.

Curiously, odd toroidal events are found to occur at a higher rate than even ones, while the opposite is true for poloidal events. This disparity may offer clues to the source of these waves, or hint at a harmonic dependence in the rate of poloidal-to-toroidal rotation.

%\todo{Do we want to get into event phase?}

%Event phase is also considered. Most events are shown to fall within
%\SI{15}{\degree} of $\pm\SI{90}{\degree}$, indicating that the traveling
%component of Pc4 pulsations is generally small compared to the standing
%component. Odd events are found to be spread more broadly in phase; this
%is likely a consequence of being measured near the equator, where (due
%to the electric field antinode) the lifetime of an odd event is
%significantly larger than that of an even event with the same phase.

% ######################################################################
% ########################################################### Discussion
% ######################################################################

\section{Discussion}

\todo{...}

% ######################################################################
% ########################################################### References
% ######################################################################

\bibliographystyle{plain}
\bibliography{bibliography.bib}

%\end{article}

% ######################################################################
% ############################################################## Figures
% ######################################################################


% ######################################################################
% ############################################################# Appendix
% ######################################################################

%\begin{appendix}
%    \label{app}
%\begin{center}
%    {\bf APPENDIX A: FBK freq and amp interpolation}
%\end{center}
%\end{appendix}

% ######################################################################
% ##################################################### Acknowledgements
% ######################################################################

\section{Acknowledgements}

%\begin{acknowledgments}

\todo{...}

%\end{acknowledgments}

% ######################################################################
% ###################################################### End of Document
% ######################################################################

%\clearpage

\end{document}

%\clearpage
