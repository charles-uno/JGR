\documentclass{article}
%\documentclass[jgrga]{agutex}


%\usepackage{natbib}
\usepackage{graphicx}
\usepackage{epstopdf}
\usepackage{times}
\usepackage{latexsym}
\usepackage{amssymb}
\usepackage[displaymath]{lineno}
\usepackage{indentfirst}
\usepackage{lineno}
\usepackage{lipsum} % For adding fake text. 

\usepackage{units} % SI unit typesetting
\usepackage{xspace} % Automatically adjusting space after macros
\usepackage{amsmath} % \text, and other math formatting options
\usepackage{siunitx} % \num{} formatting and SI unit formatting

\usepackage{parskip} % http://ctan.org/pkg/parskip vskip instead of indent. 

\usepackage[noabbrev,capitalize]{cleveref} % Automatically determine \cref type

\usepackage{xcolor} % so we can put todo notes in color. 

% Configure the siunitx package
\sisetup{
    group-separator = {,}, % Use , to separate groups of digits, like 12,345
    list-final-separator = {, and } % Always use the serial comma in \SIlist
}

% Configure the cleveref package
\newcommand{\creflastconjunction}{, and } % Always use the serial comma

\linenumbers*[1]

%use to display figures in draft mode
\setkeys{Gin}{draft=false}


% ######################################################################
% ################################################### Shorthand / Macros
% ######################################################################

\newcommand{\about}{\ensuremath{\sim}}

\newcommand{\real}{\ensuremath{\mathbb{R}\mathrm{e}}\xspace}
\newcommand{\imag}{\ensuremath{\mathbb{I}\mathrm{m}}\xspace}

\newcommand{\DST}{\text{Dst}\xspace}

\newcommand{\dft}[1]{\ensuremath{\overset{\sim}{#1}}\xspace}

% show todo notes in red. 
\newcommand{\todo}[1]{ \textcolor{red}{TODO: #1} }
% hide todo notes. 
%\newcommand{\todo}[1]{}

% Names with special characters. 
\newcommand{\Alfven}{Alfv\'en\xspace}
\newcommand{\Ampere}{Amp\`ere\xspace}
\newcommand{\Schrodinger}{Schr\"odinger\xspace}

% To make sure the capitalization is consistent. 
\newcommand{\ohmlaw}{Ohm's law\xspace}
\newcommand{\amplaw}{\Ampere's law\xspace}
\newcommand{\farlaw}{Faraday's law\xspace}
\newcommand{\maxeqs}{Maxwell's equations\xspace}

% What should Radoski's dipole coordinates be named? \nu is overused. 
\newcommand{\radx}{\ensuremath{x}\xspace}
\newcommand{\rady}{\ensuremath{y}\xspace}
\newcommand{\radz}{\ensuremath{z}\xspace}

% What should Lysak's coordinates be named? I don't love u1 u2 u3. 
\newcommand{\lysaki}{\ensuremath{u^i}\xspace}
\newcommand{\lysakj}{\ensuremath{u^j}\xspace}
\newcommand{\lysakx}{\ensuremath{u^1}\xspace}
\newcommand{\lysaky}{\ensuremath{u^2}\xspace}
\newcommand{\lysakz}{\ensuremath{u^3}\xspace}

% Coordinate names... these should probably be italicized? 
\newcommand{\x}{\ensuremath{x}\xspace}
\newcommand{\y}{\ensuremath{y}\xspace}
\newcommand{\z}{\ensuremath{z}\xspace}
\newcommand{\X}{\ensuremath{X}\xspace}
\newcommand{\Y}{\ensuremath{Y}\xspace}
\newcommand{\Z}{\ensuremath{Z}\xspace}

% Field-aligned unit vectors. 
\newcommand{\ehat}{\ensuremath{\hat{e}}\xspace}
\newcommand{\xhat}{\ensuremath{\hat{x}}\xspace}
\newcommand{\yhat}{\ensuremath{\hat{y}}\xspace}
\newcommand{\zhat}{\ensuremath{\hat{z}}\xspace}
\newcommand{\Xhat}{\ensuremath{\hat{X}}\xspace}
\newcommand{\Yhat}{\ensuremath{\hat{Y}}\xspace}
\newcommand{\Zhat}{\ensuremath{\hat{Z}}\xspace}

% Spherical unit vectors. 
\newcommand{\rhat}{\ensuremath{\hat{r}}\xspace}
\newcommand{\qhat}{\ensuremath{\hat{\theta}}\xspace}
\newcommand{\fhat}{\ensuremath{\hat{\phi}}\xspace}

% Use underlines for vectors and tensors. 
\renewcommand{\vec}[1]{\ensuremath{\underline{#1}}}
\newcommand{\tensor}[1]{\ensuremath{\underline{\underline{#1}}}}

% Differential operators. 
\newcommand{\dd}[1]{\ensuremath{ \frac{\partial}{\partial #1} }\xspace}
\newcommand{\ddt}{\dd{t}\xspace}
\newcommand{\curl}[1]{\ensuremath{ \nabla \times \vec{#1} }\xspace}
\renewcommand{\div}[1]{\ensuremath{ \nabla \cdot \vec{#1} }\xspace}
\newcommand{\grad}[1]{\ensuremath{ \nabla #1 }\xspace}

% Properly-scaled parentheses for grouping terms or for arguments. 
\newcommand{\lr}[1]{ \left( #1 \right) }
\newcommand{\lrsmall}[1]{ \left( {\scriptstyle #1} \right) }
\renewcommand{\arg}[1]{\!\lr{#1}}
\newcommand{\argsmall}[1]{\!\lrsmall{#1}}
\newcommand{\lrb}[1]{ \left[ #1 \right] }
\newcommand{\argb}[1]{\!\lrb{#1}}

% Define a better looking eV by moving the V slightly left
\DeclareSIUnit\electronvolt{e\hspace{-0.08em}V}
\DeclareSIUnit\keV{\kilo\electronvolt}
\DeclareSIUnit\percc{/\cm\cubed}
\DeclareSIUnit\RE{R_E}
\DeclareSIUnit\nT{\nano\tesla}
\DeclareSIUnit\nJ{\nano\joule}
\DeclareSIUnit\S{S}
\DeclareSIUnit\Mm{Mm}

\newcommand{\dt}{\ensuremath{\delta \hspace{-0.1em} t}\xspace}
\newcommand{\dr}{\ensuremath{\delta \hspace{-0.1em} r}\xspace}
\newcommand{\dL}{\ensuremath{\delta \hspace{-0.1em} L}\xspace}

% Assignment operator for pseudocode. 
\newcommand{\assign}{\ensuremath{\leftarrow}\xspace}

% Azimuthal modenumber, typically indicated with a lowercase m. 
\newcommand{\azm}{\ensuremath{m}\xspace}

% Electron mass, also typically indicated with a lowercase m. 
\newcommand{\me}{\ensuremath{m_{e}}\xspace}

% Jacobian dererminant, typically indicated with a capital J, which we are using for current. 
\newcommand{\jac}{\ensuremath{ \sqrt{g} }\xspace}

% Plasma frequency
\newcommand{\op}{\ensuremath{\omega_P}\xspace}

% Alfven speed. 
\newcommand{\va}{\ensuremath{v_A}\xspace}

% Perpendicular electric constant. 
\newcommand{\ep}{\ensuremath{\epsilon_\bot}\xspace}

% Epsilon zero, mu zero, and one over mu zero. 
\newcommand{\ez}{\ensuremath{\epsilon_0}\xspace}
\newcommand{\mz}{\ensuremath{\mu_0}\xspace}
\newcommand{\oomz}{\ensuremath{ \frac{1}{\mz} }\xspace}

% Conductivities. 
\newcommand{\sz}{\ensuremath{\sigma_0}\xspace}
\newcommand{\sh}{\ensuremath{\sigma_H}\xspace}
\renewcommand{\sp}{\ensuremath{\sigma_P}\xspace}

% Speed of light. 
\newcommand{\C}{{\mathrm{c}}}

% Add space between rows of tables
\newcommand{\spacerows}[1]{\renewcommand{\arraystretch}{#1}}

% ######################################################################
% ################################################# Title, Abstract, etc
% ######################################################################

\begin{document}
\title{Modeling Pc4 Pulsations in Two and a Half Dimensions with
       Comparisons to Van Allen Probes Observations}
\author{
    Charles McEachern\textsuperscript{1},
    Robert Lysak\textsuperscript{1},
    ... \\
    \textsuperscript{1} University of Minnesota
}

%\affil{University of Minnesota}

%\lefthead{McEachern et al.}
%\righthead{Pc4s in 2.5D}
%\linespread{2}

%\linenumbers

\maketitle


% ======================================================================
% ============================================================= Abstract
% ======================================================================

\begin{abstract}

Field line resonances in the Pc4 range (\SIrange{7}{25}{\mHz}) serve
to energize magnetospheric particles through drift-resonant
interactions, carry energy from high to low altitude, induce currents in
the magnetosphere, and accelerate particles into the atmosphere. Wave
structure and polarization significantly impact the execution these
roles. The present work showcases a new two and a half dimensional code,
Tuna, ideally suited to model FLRs, with the ability to consider
large-but-finite azimuthal modenumbers, coupling between the poloidal,
toroidal, and compressional modes, and arbitrary harmonic structure.
Using Tuna, the interplay between Joule dissipation and
poloidal-to-toroidal rotation is considered for Pc4 pulsations under
both dayside and nightside conditions. An attempt is also made to
demystify giant pulsations, a class of Pc4 noted for its distinctive
ground signatures. Numerical results are supplemented by a survey of
\about700 Pc4 pulsations using data from the Van Allen Probes, the first
such survey to characterize each event by both polarization and
harmonic. The combination of numerical and observational results
suggests an explanation for the disparate distributions observed in
poloidal and toroidal Pc4 events. 

\end{abstract}

%\begin{article}

% ######################################################################
% ######################################################### Introduction
% ######################################################################

\section{Introduction}

Pc4s are interesting. 

\begin{itemize}
    \item Drift-resonant interactions with trapped energetic particles. 
    \item Radial diffusion. 
    \item Giant pulsations are mysterious and exciting. 
\end{itemize}

Pc4s can be classified in terms of their harmonic numbers. 

\begin{itemize}
    \item First harmonic is for drift resonance. 
    \item Second harmonic is for drift-bounce resonance. 
    \item You can sorta tell them apart by frequency, but there are disagreements. Mostly you need two observations -- satellite plus ground, or E and B on the same machine, like the Van Allen Probes. 
\end{itemize}

They can also be classified in terms of their azimuthal modenumber. 

\begin{itemize}
    \item One wavelength of a wave with modenumber \azm spans $\frac{24}{\azm}$ hours in MLT. 
    \item Small \azm typically driven by broadband solar wind behavior. The waves are compressional -- they can move across magnetic field lines. 
    \item Large \azm is driven by resonant interactions with trapped particles -- the waves are not compressional, so propagation is evanescent across field lines. 
    \item Ground observations depend on \azm. At large \azm, a wave's ground signature is attenuated by the ionosphere. 
\end{itemize}

They can also be classified in terms of their polarization. 

\begin{itemize}
    \item Poloidal waves pulse. Magnetic field lines perturb radially. Associated azimuthal electric field perturbations. 
    \item Toroidal waves twist. Magnetic field lines perturb azimuthally. Associated radial electric field perturbations. 
    \item Poloidal waves rotate asymptotically to the toroidal mode, per \cite{mann_1995}. High-\azm poloidal waves have shorter lifetimes. 
    \item Most observed waves are toroidal. 
    \item Most poloidal waves are even modes. 
    \item Toroidal frequencies are sharply defined in L. Poloidal show a more smeared-out dependence. \cite{engebretson_1986}
    \item In this regime, wave polarization is rotated about \SI{90}{\degree} by the ionosphere\cite{}. An east-west magnetic perturbation in space is observed as a north-south perturbation at Earth's surface. 
\end{itemize}

Let's also talk about giant pulsations. 

\begin{itemize}
    \item Strong, well-formed ground signatures. Tabulated by eye for a century. 
    \item Odd poloidal Pc4s, peaked from postmidnight to the early morning. 
    \item A distinct break from mostly-toroidal, mostly-dayside Pc4s in general. 
\end{itemize}

Past models of the magnetosphere have been limited in their
consideration of FLRs. Reasons include overly-simplified treatment of
the ionospheric boundary, no consideration of the plasmapause, limited
range in \azm, and the inability to compute ground signatures. The
present work showcases a model which addresses these issues, providing a
bird’s-eye view of the structure and evolution of FLRs. 

Using this model, the present work explores novel connections between
several of the seemingly-disparate Pc4 properties listed above.
Poloidal-to-toroidal rotation timescales are compared to dayside and
nightside Joule dissipation timescales; the implications to Pc4
observations are considered. The strength and structure of ground
signatures is investigated as function of ionospheric and driving
conditions. And the distinctive properties associated with giant
pulsations are matched against those of fundamental poloidal Pc4s in
general. 

Results are then validated against a survey of \about700 Pc4
observations using data collected by the Van Allen probes. In contrast
to other ULF wave surveys (for example, \cite{dai_2015} and
\cite{motoba_2015}), the present work classifies each event by both
polarization and harmonic. This crucial aspect of the analysis is
possible only because the Van Allen Probes measure both electric and
magnetic field waveforms. No past mission has provided access to such a
rich data set for ULF wave events in the inner magnetosphere. 

% ######################################################################
% ################################################################ Model
% ######################################################################

\section{Numerical Model}

\todo{You can get this code on GitHub!}

Numerical results are obtained using Tuna, a new linear \Alfven wave
code based on that described in \cite{lysak_2013}. Tuna models the
evolution of three-dimensional electric and magnetic fields over a
(two-dimensional) meridional slice of the magnetosphere. The code can
colloquially be said to have two and a half (``tuna half'') dimensions,
hence the name. 

For the purpose of evaluating derivatives in the azimuthal direction,
fields are taken to vary as $\exp \arg{i \azm \phi}$ for fixed azimuthal
modenumber \azm. Azimuthal derivatives are replaced by a factor of
$i \azm$; fields are complex-valued as a result. This assumption is
easily justified in the case of Pc4 pulsations, which are typically
localized in MLT, per
\cite{anderson_1990,dai_2015,engebretson_1992,liu_2009}. 

% ======================================================================
% ========================================== Physical Parameter Profiles
% ======================================================================

Empirical profiles are used for the conductivity tensor
$\tensor{\sigma}$ and the electric tensor $\tensor{\epsilon}$. The
conductivity tensor $\tensor{\sigma}$ comes from values tabulated in
\cite{kelley_1989}, and modified per \cite{lysak_2013} to take into
account the loading of oxygen ions near the atmosphere. The electric
tensor $\tensor{\epsilon}$ gives characteristic velocity $c$ along the
magnetic field line and $\va$ in the perpendicular direction, where the
\Alfven speed $\va$ is defined per
\begin{align}
    \label{def_va}
    \va^2 &\equiv \frac{B^2}{\rho} &
    & \text{or, equally,} &
    \va^2 &\equiv \frac{1}{\mz\ep} 
\end{align}

\todo{Figure of conductivity profiles? Dissertation figure 5.3}

In \cref{def_va}, $B$ is the magnitude of the zeroth-order magnetic
field, taken to be an ideal dipole field with magnitude \SI{3.1e4}{\nT}
at the equator at Earth's surface. The density profile is modeled as the
sum of a latitude-independent profile (\SI{10}{\percc} at the
ionosphere, falling off as $\frac{1}{r}$) and a latitude-dependent one
(\SI{e4}{\percc} at the ionosphere, with a sharp drop at $L = 4$).

Four different physical parameter profiles are used for conductivity and
\Alfven speed, corresponding to dayside and nightside conditions at the
top and bottom of the solar cycle. 

% ======================================================================
% ============================================ Nonorthogonal Dipole Grid
% ======================================================================

Tuna's grid follows the modified dipole coordinates described in
\cite{lysak_2004}:
\begin{align}
  \label{def_coords}
  \lysakx & = - \frac{R}{r} \sin^2 \theta & 
  \lysaky & = \phi &
  \lysakz & = \frac{R^2}{r^2} \frac{\cos \theta}{\cos \theta_0}
\end{align}

Here, $R$ is the geocentric radius of the ionospheric boundary, taken to
be at $R_E + \SI{100}{\km}$, $\theta_0$ is the invariant colatitude, and
$r$, $\theta$, and $\phi$ are the usual spherical coordinates. 

\todo{Figure of the grid? Dissertation figure 5.1}

A thorough discussion of these coordinates, and explicit forms for the
resulting basis vectors and metric tensor components can be found in
\cite{lysak_2004}. At present, it's sufficient to note that the
coordinates are nonorthogonal, and thus have covariant and contravariant
basis vectors (${\ehat_i \equiv \dd{\lysaki}\vec{r}}$ and
${\ehat^i \equiv \dd{\vec{r}}\lysaki}$ respectively) that do not line up
with one another --- but that both the covariant and contravariant bases
are valuable. 

The basis vectors $\ehat^1$, $\ehat^2$, and $\ehat_3$ provide a mapping
to the usual dipole coordinates, as shown in \cref{to_dipole}, which is
the natural basis for the conductivity and electric tensors. 
\begin{align}
    \label{to_dipole}
    \ehat^1 &\parallel \xhat &
    \ehat^2 &\parallel \yhat &
    \ehat_3 &\parallel \zhat
\end{align}

Where $\zhat$ lies along the magnetic field, $\yhat$ is azimuthally
eastward, and $\xhat \equiv \yhat \times \zhat$ points radially outward
at the equator. 

In addition, at the ionospheric boundary, $\ehat_1$, $\ehat_2$, and
$\ehat^3$ map to the spherical basis:
\begin{align}
  \ehat_1 &\parallel \hat{\theta} &
  \ehat_2 &\parallel \hat{\phi} &
  \ehat^3 &\parallel \hat{r}
\end{align}

As a result, Tuna's grid is aligned everywhere to the zeroth-order
dipole magnetic field, while also supporting a fixed-altitude
ionospheric boundary. 

The results shown in the present work use a grid of 150 values in 
\lysakx (150 field lines) and 350 values in \lysakz (350 grid points per
field line). Spacing is on the order of \SI{10}{\km} near the ionosphere
and \SI{1000}{\km} at the equator of the outermost field line. The inner
boundary is placed at $L = 2$, and the outer boundary at $L = 10$. The
time step is determined from the smallest \Alfven crossing time, scaled
down by a Courant factor of \num{0.1}. Typically,
$\dt \about \SI{10}{\us}$.

% ======================================================================
% ================================================= Ring Current Driving
% ======================================================================

Like the similar models of \cite{lysak_2013} and \cite{waters_2013},
Tuna can be driven via compression of the simulation's outer boundary --
typically taken as a proxy for solar wind activity. However, the shear
and compressional \Alfven modes decouple at high \azm, preventing such
waves from propagating across magnetic field lines. In order to model
noncompressional Pc4 activity at $L\about5$, it's necessary to inject
energy at $L\about5$. 

To this end, Tuna also allows simulations to be driven via modulation of
the ring current, a stand-in for substorm injection events. The driving
current is spread over a cross section of \about\SI{1}{\RE\squared},
centered just outside the plasmapause and just off the equator. 

The magnitude of the driving current is estimated from a discrete
Fourier transform of the Sym-H storm index during the June 2013 storm.
Sym-H is tabulated once per minute, which is too slow to capture
activity in the Pc4 band directly. However, a fit of the pink noise
suggests that activity in the Pc4 range could plausibly give rise to a
field at Earth's surface on the order of \SI{e-2}{\nT}. 

This corresponds to a ring current perturbation on the order of
\SI{1}{\mega\A}, or \about\SI{e-4}{\uA/\m\squared} spread over
\about\SI{1}{\RE\squared}. For the runs shown in the present work, the
driving current is azimuthally directed, and its magnitude varies
sinusoidally. 

% ======================================================================
% ================================================== Maxwell's Equations
% ======================================================================

Mathematically, the driving current is introduced through an anomalous
current term $\vec{J}$ in \Ampere's law, apart from the Ohmic current
${\tensor{\sigma} \cdot \vec{E}}$.
\begin{align}
    \label{amp_law}
    \tensor{\epsilon} \cdot \ddt \vec{E} &= \oomz \curl{B} - \vec{J}
      - \tensor{\sigma} \cdot \vec{E}
\end{align}

Because of the Ohmic current term, the electric field's time derivative
in \cref{amp_law} depends on its own future value. This circular
dependence is resolved using integrating factors. First, the expression
is rewritten:
\begin{align}
    \label{int_fac}
    \Big( \tensor{\Omega} + \tensor{ \mathbb{I} }\ddt \Big) \cdot
        \vec{E} &= \tensor{V}^2 \cdot \vec{F}
\end{align}

Where $\tensor{ \mathbb{I} }$ is the identity and $\vec{F}$,
$\tensor{V}^2$, and $\tensor{\Omega}$ are shorthand:
\begin{align}
    \vec{F} &\equiv \curl{B} - \mz \vec{J} &
    \tensor{V}^2 &\equiv \frac{1}{\mz} \tensor{\epsilon}^{-1} &
    \tensor{\Omega} &\equiv \tensor{\epsilon}^{-1} \cdot \tensor{\sigma}
\end{align}

\cref{int_fac} is then solved by multiplying through by
$\exp \arg{ \tensor{\Omega} \, t }$ (see \cite{hall_2015}), applying the
product rule, and integrating over a time step \dt.
\begin{align}
    \label{amp_final}
    \vec{E} &\assign \exp \arg{ -\tensor{\Omega} \, \dt } \cdot \vec{E} +
      \dt \, \exp \arg{ -\tensor{\Omega} \, \tfrac{\dt}{2} } \cdot
      \tensor{V}^2 \cdot \vec{F}
\end{align}

\cref{amp_final} is evaluated by separating the exponential into its
diagonal (Pedersen) and off-diagonal (Hall) terms. The Hall terms give a
rotation matrix around the magnetic field line, coupling the poloidal
and toroidal modes, consistent with \cite{hughes_1974}. Terms
proportional to $\exp \arg{ - \frac{\sz}{\ez}\dt }$ are also present.
However, $\frac{\sz}{\ez}\dt \gtrsim 1000$, so these terms are
vanishingly small. As a result, parallel electric fields are taken to be
uniformly zero. 

Magnetic fields are simply advanced using Faraday's law:
\begin{align}
    \label{far_law}
    \ddt \vec{B} &= - \curl{E}
\end{align}

For the sake of brevity, the present work does not expand the terms of
\cref{amp_final,far_law} in the covariant basis. Those expressions can
be found in \cite{mceachern_2016}. 

% ======================================================================
% ================================================== Boundary Conditions
% ======================================================================

\todo{Dirichlet and Neumann conditions are applied to the electric and magnetic fields respectively (?) at the inner and outer boundaries. In practive, most wave activity is concentrated far from the boundary, so this doesn't matter much. Results are robust under an exchange of the two.}

\todo{At the ionospheric boundary, say there's a thin current sheet representing the E layer of the ionosphere. Below that, the atmosphere is perfectly insulating, $\curl{B} = 0$. Along with $\div{B} = 0$ per Maxwell's equations, that gives the magnetic field in terms of a scalar magnetic potential $\Psi$, $\vec{B} = \nabla\Psi$ where $\nabla^2 \Psi = 0$. }

\todo{Analytically, the solution is spherical harmonics. But we do a numerical solution instead, to ensure orthonormality on the incomplete grid (no pole or equator). }

\todo{Assume that the thin current sheet at the top of the atmosphere is perfectly horizontal. Then he vertical component of the magnetic field is continuous through it. }

\todo{Assume that Earth is a perfect conductor, so the vertical magnetic field goes to zero at the surface.}

\todo{Finagle some things. We get some coefficients that take us from the vertical magnetic field to the $\theta$ and $\phi$ components of the magnetic field at the top and bottom of the atmosphere. The values at the top get plugged back in as boundary conditions; the ones are the bottom are purely output, and can be compared to ground magnetometer data.}

\todo{See \cite{fujita_1988,lysak_2013}. }

% ######################################################################
% #################################################### Numerical Results
% ######################################################################

\section{Numerical Results}

\todo{Figure: snapshots of poloidal, toroidal, compressional magnetic field after 300 seconds, high \azm versus low \azm. Just dayside should be fine. }

\todo{Figure: integrated energy in the poloidal and toroidal modes as a function of time. Analogous to figure 3 in \cite{mann_1995}. Probably two rows, one for dayside and one for nightside, rather than two figures. Four across should do it --- \azm of 1, 4, 16, 64. }

\todo{Figure: Energy distribution by $L$-shell. Again, \azm of 1, 4, 16, 64. Probably four rows, for dayside/nightside poloidal/toroidal. }

\todo{Figure: Ground signatures. The best peaks are at \azm of 16 and 32, but what we're really showing is that midsize \azm are stronger than those at very small or very large values. So should work to again show 1, 4, 16, 64. And four rows, dayside/nightside $B_\phi$/$B_\theta$. Note that $B_\phi$ is actually poloidal because of the \about\SI{90}{\degree} rotation due to the ionosphere. }


The above results show agreement with a number of past FLR studies. In
addition, several novel connections are suggested between known
properties of Pc4 pulsations. 

The present results suggest that the compressional character of poloidal
Pc4s is to blame for the weak relationship they exhibit between $L$ and
frequency, compared to that seen in toroidal events. Toroidal resonances
are defined sharply in $L$ regardless of modenumber, while poloidal
resonances are smeared in $L$ --- particularly at low \azm, but to some
degree at high \azm as well. 

The asymptotic rotation of energy from the poloidal mode to the toroidal
mode is reproduced; at small \azm, the rotation timescale is comparable
to a wave period, while at large modenumber it's on the order of 10
periods. On the dayside, little energy is lost to Joule dissipation on
rotation timescales, suggesting that the poloidal mode is a significant
source of energy for same-harmonic toroidal waves. On the nightside,
dissipation timescales are comparable to wave periods, suggesting that
the poloidal mode gives rise to toroidal waves less effectively. 

Ground signatures at low modenumber are shown to be weak because waves
in space are weak; this is particularly true for poloidal waves --- a
non-guided wave can't very well resonate along a field line --- but also
true of toroidal waves insomuch as poloidal waves are their source. Pc4s
resonate most strongly at high \azm, but high-\azm signatures are also
attenuated by the atmosphere. The balance between the two effects falls
around \azm of 16 to 32. It's further suggested that a high-\azm driver
will cause a weak resonance in place rather than tunneling across field
lines to a matching eigenfrequency, and that the same driving should
give rise to stronger ground signatures during times of low solar
activity, on both the dayside and the nightside. 

\todo{Do we actually want to get into quiet versus active? Doubtful.}

The findings together suggest, awkwardly, that the morphology of giant
pulsations reveals relatively little about their origins. 

One can consider a hypothetical magnetosphere subject to constant
driving: broadband in frequency, broadband in modenumber, just outside
the plasmapause. Low-\azm poloidal waves will quickly rotate to the
toroidal mode (and/or propagate away). High-\azm waves will resonate in
place, accumulating energy over time, and giving rise to ``multiharmonic
toroidal waves'' (per \cite{takahashi_2011}); Fourier components that do
not match the local eigenfrequency will accumulate energy over just a
few wave periods before reaching asymptotic values. Waves with very high
modenumbers will be attenuated by the ionosphere. The response on the
ground will be counterclockwise at low latitude, clockwise at high
latitude, peaked at $16 \lesssim \azm \lesssim 32$, mostly east-west
polarized, and notably stronger during quiet solar conditions. In other
words, the measurements will look very much like a giant pulsation. 

The present results offer no explanation as to the tendency of giant
pulsations to drift azimuthally, or to appear pre-dawn in MLT --- though
the latter is addressed by the observational results in below. 

% ######################################################################
% ##################################################### Van Allen Probes
% ######################################################################

\section{Van Allen Probes Observations}

\todo{Figure: A sample event.}

\todo{Figure: Distribution of all Pc4 events.}

\todo{Figure: Distribution of Pc4 events by mode.}

The present chapter gives a survey of \about700 thirty-minute Pc4
events, each characterized in terms of both parity and polarization, and
selected in a way that does not introduce an apparent bias in either
property. No past study has so thoroughly disentangled the parity and
polarization of these waves.

Coarsely speaking, event distributions are found to be consistent with
past surveys. Toroidal events dominate overall, and are primarily seen
on the morning side. Poloidal events are spread broadly in MLT, with a
peak near noon and distinctive odd harmonics in the early morning. From
there, the simultaneous consideration of harmonic and polarization,
combined with the numerical results above, offers significant insight. 

The near-noon peak of poloidal Pc4 events is shown to be due to even
events (a majority subset). Odd poloidal events occur preferrentially
near midnight and across the morning side. Similarly, toroidal events
are mostly odd, and it is specifically the odd toroidal events which
occur on the morningside, while even toroidal events peak near noon. 

The spatial distribution of even poloidal events looks much like the
spatial distribution of even toroidal events, except that the toroidal
distribution is skewed dayward compared to the poloidal. The same can be
said of the odd events. This is consistent (per the above numerical
results) with poloidal events as an effective source for (same-parity)
toroidal events on the dayside, and a less-effective source on the
nightside. 

As a corrolary, the distribution of odd poloidal events is found to
closely resemble the distribution of giant pulsations: midnight and
morning. This (consistent with the numerical results) suggests that the
distinctive properties attributed to giant pulsations are in fact shared
by odd poloidal Pc4s overall. 

Curiously, odd toroidal events are found to occur at a higher rate than
even ones, while the opposite is true for poloidal events. This
disparity may offer clues to the source of these waves, or hint at a
harmonic dependence in the rate of poloidal-to-toroidal rotation. 

\todo{Do we want to get into event phase?}

Event phase is also considered. Most events are shown to fall within
\SI{15}{\degree} of $\pm\SI{90}{\degree}$, indicating that the traveling
component of Pc4 pulsations is generally small compared to the standing
component. Odd events are found to be spread more broadly in phase; this
is likely a consequence of being measured near the equator, where (due
to the electric field antinode) the lifetime of an odd event is
significantly larger than that of an even event with the same phase. 

% ######################################################################
% ########################################################### Discussion
% ######################################################################

\section{Discussion}

\todo{...}

% ######################################################################
% ########################################################### References
% ######################################################################

\bibliographystyle{plain}
\bibliography{bibliography.bib}

%\end{article}

% ######################################################################
% ############################################################## Figures
% ######################################################################

\begin{figure}
    \label{fig_placeholder_wide}
    \begin{center}
    \includegraphics[width=\textwidth]{figures/placeholder.jpg}
    \caption{
        A wide placeholder figure. Note that Minerva looks very stern.
    }
    \end{center}
\end{figure}



% ######################################################################
% ############################################################# Appendix
% ######################################################################

%\begin{appendix}
%    \label{app}
%\begin{center}
%    {\bf APPENDIX A: FBK freq and amp interpolation}
%\end{center}
%\end{appendix}

% ######################################################################
% ##################################################### Acknowledgements
% ######################################################################

\section{Acknowledgements}

%\begin{acknowledgments}

\lipsum[1]

%\end{acknowledgments}

% ######################################################################
% ###################################################### End of Document
% ######################################################################

%\clearpage

\end{document}

%\clearpage


